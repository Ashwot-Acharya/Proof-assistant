\documentclass{article}
\author{ Ashwot Acharya, Bishesh Bohora, Supreme chaudary}

\title{Working Principles Of Proof Assistants And Formalization Of Some Proofs In Linear Algebra}

\begin{document}
\maketitle
\pagebreak
\section{Introduction}
Proof assistants helps with formalization of mathematical proofs in computer which enables for faster proofs computations.
Some proofs could have many many special cases which isn't easy to figureout using tradational methods, but with the use of proof assistant these cases can easily be accounted for.
Proof assistant aren't just about guiding with complicated proofs but also can help to check if the proof that is written is correct especially for those long and tedious proofs.

In 1998 Thomas C. Hales announced that he had proved the kepler conjecture in 1998 (Solane,1998) However the peer review took over 4 years especially due to the proof being incredibely difficult to check , with over a dozen of mathematician to refree the proof and although 
they were 99\% certain it was not untill the proof became formalized that they were able to completely validate the correctness of the proof, which according to Thomas Hales would have taken 20 person-year of manual work.(Szpiro, G, 2003)


\section{Problem statement}
This projects aims to design and implementat Haskell-Based proof assistant software and formalize some  of the proofs of linear Algebra. Leveraging Haskell's strong type system and its purely functional nature.
\section{Research objectives}
\begin{enumerate}
    \item To Investigate the current exsisting proof assistant software and identify the mathematics behind them.
    \item To implementat Intuitionistic type theory and apply the logic of constructivism.
    \item To formalize some selected proofs of Linear Algebra
\end{enumerate}
\section{Literature Review}

\section{Methodology}

\section{Expected Outcomes}
\section{Significiance}
\section{Work Plan}
\pagebreak
\section{Refrences}
\begin{enumerate}
    \item Sloane, N.J.A. (1998). Kepler's conjecture confirmed. Nature, 395(6701), pp.435-436 doi:https://doi.org/10.1038/26609.
    \item Szpiro, G. (2003). Does the proof stack up? Nature, 424(6944), pp.12–13,  doi:https://doi.org/10.1038/424012a.


\end{enumerate} 

\end{document}