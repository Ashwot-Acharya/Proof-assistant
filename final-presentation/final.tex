\documentclass[aspectratio=169, compress, 12pt]{beamer}

\usepackage[english]{babel}                 %%%%%%%%%%%%%%%%%% (EDIT LANGUAGE)

\usepackage{Custom}                         %%%%%%%%%%%%%%%%%%    (EDIT NAMES)
\usepackage{Custom-2}                         %%%%%%%%%%%%%%%%%%    (EDIT COLOURS)
\usepackage{AuxiliaryFiles/AuxiliaryFiles}  %%%%%%%%%%%%%    (DO NOT EDIT)

\begin{document}
\TitlePage        
\SectionPage    
\SubsectionPage    %%%% PAGE AT THE BEGINNIN OF A SECTION - IF NOT NEEDED, COMMENT IT
\ProgressBar       %%%% PROGRESS BAR ON THE RIGHT MARGIN - IF NOT NEEDED, COMMENT IT
\PageNumbering     %%%% SLIDES NUMBERING - IF NOT NEEDED, COMMENT IT


\input{AuxiliaryFiles/SettingBackup}     

\section{Proof Assistants} 
\subsection*{What are proof assistant}
\begin{frame}{What are proof assistant}
    proof assistant, are software more specifically a type of programming language thats allows us to 
    formalize mathematical proofs in computer for digital verification. 
\end{frame}
\subsection*{Why digital verification is needed?}
\begin{frame}{Need of digital verification}
    \begin{itemize}
        \item faster computation for complex problems
        \item many exceptional cases can be explored which would take mathematicians  long time \\
           \indent  \footnotesize ex:   The Kepler Conjecture's proof , which was so complex that verifying it manually would take 20 person-years, but proof assistants made this verification feasible and fast. 
    \end{itemize}
\end{frame}
\section{Logical foundation}
\begin{frame}
    \begin{itemize}
    \item Based on logic (natural deduction, intuitionistic logic), $\lambda$-calculus, and type theory.
    \item Curry–Howard Correspondence:
    \begin{itemize}
        \item Propositions $\leftrightarrow$ Types
        \item Proofs $\leftrightarrow$ Programs
    \end{itemize}
    \item Dependently Typed Languages: Types can depend on values, enabling encoding of properties and proofs.
\end{itemize}
\end{frame}
\subsection*{Naturl deduction}
\begin{frame}{Natural Deduction}
\begin{itemize}
    \item \textbf{Natural Deduction} is a rule-based system for deriving conclusions from assumptions in logic.
    \item Instead of using exhaustive truth tables, proofs are built step-by-step using inference rules.
    \item Example: Proving from $A \land (A \rightarrow \bot)$ that $\bot$ (contradiction) can be derived.
    \item Basis for how proof assistants check the logical structure of proofs.
\end{itemize}
\end{frame}
\subsection*{Ins}
\begin{frame}{Intuitionistic Logic}
\begin{itemize}
    \item \textbf{Intuitionistic Logic} formalizes constructive mathematics, where a statement is only true if a proof can be constructed.
    \item Omits some classical logic principles, such as the Law of Excluded Middle.
    \item Stronger requirement: to prove existence, a method or algorithm must be given.
    \item Proof assistants leverage this constructive approach for digital verification.
\end{itemize}
\end{frame}
\subsection*{$\lambda$-Calculus}
\begin{frame}{$\lambda$-Calculus and Type Theory}
\begin{itemize}
    \item \textbf{$\lambda$-Calculus}: A foundational system for defining and applying functions using abstraction and application.
    \item \textbf{Type Theory}: Assigns types to every term; ensures correctness of operations.
    \item \textbf{Curry--Howard Correspondence}: 
    \begin{itemize}
        \item Propositions $\leftrightarrow$ Types
        \item Proofs $\leftrightarrow$ Programs
    \end{itemize}
    \item \emph{Dependent types} allow types to depend on values, expressing complex logical properties.
\end{itemize}
\end{frame}
\section{Architecture of proof assistant}
\begin{frame}{Architecture of a Proof Assistant}
\begin{itemize}
    \item \textbf{Kernel}: Minimal, trustworthy codebase enforcing logical rules and validating proofs.
    \item \textbf{Tactic Engine}: Helps build and automate proofs step by step.
    \item \textbf{Formal Proof Language}: Rigorously expresses definitions, statements, and proofs.
    \item \textbf{Libraries}: Collections of verified mathematical foundations for reuse.
    \item \textbf{User Interface}: IDEs and plugins for interactive, efficient proof development.
\end{itemize}
\end{frame}
\section{Comparative Study}
\subsection*{Agda}
\begin{frame}{Agda}
\begin{itemize}
    \item \textbf{Dependently typed functional programming language and proof assistant}.
    \item No separate tactic layer---proofs are constructed explicitly and directly.
    \item Uses \textbf{bidirectional type checking}, minimal kernel, and integrates normalization.
    \item Emphasizes explicitness and formal clarity over automation.
    \item Best suited for foundational work and educational purposes.
\end{itemize}
\end{frame}
\subsection*{Rocq (Coq)}
\begin{frame}{Rocq (Coq)}
\begin{itemize}
    \item \textbf{Tactic-based interactive theorem prover} famous for large formalizations (e.g., Four Color Theorem).
    \item Kernel is based on Calculus of Inductive Constructions (CIC), written in Coq and extracted to OCaml.
    \item Proofs often built with backward reasoning using tactics---automation is heavily supported.
    \item Strong type-checking with bidirectional algorithms and conversion checking.
    \item Widely used for complex, real-world formal verification tasks.
\end{itemize}
\end{frame}
\subsection*{Lean}
\begin{frame}{Lean}
\begin{itemize}
    \item Developed at Microsoft, supports both \textbf{tactic-based and term-based} proofs.
    \item Kernel based on the Calculus of Inductive Constructions, implemented in C++/C.
    \item Allows seamless switching between tactic and term styles---user-friendly and flexible.
    \item Smart elaborator with coercion, backtracking, and overloading.
    \item Popular in research and education, especially for combinatorial and mathematical formalizations.
\end{itemize}
\end{frame}
\section{Formalization Of Some proofs}


% !TEX root = ../BeamerTemplate.tex
%%%%%%%%%%%%%%%%%%%%%%%%%%%%%%%%%%%%%%%%%%%%%%%%%%%%%%%%%%%%%%%%%%%%%%%%%%%%%%%%%%
{\setbeamercolor{background canvas}{bg=Blue}
\begingroup
\setbeamertemplate{navigation symbols}{}
\begin{frame}[plain]\label{Thanks}
    \bf\Huge\color{white}
    \begin{center}
            Thank you!

    \end{center}
\end{frame}
\endgroup
}  
\end{document}