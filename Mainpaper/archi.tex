% proof_assistants_architecture.tex
\documentclass[11pt,a4paper]{article}
\usepackage{amsmath,amssymb,url,hyperref,geometry,xcolor}
\geometry{margin=2.5cm}
\newcommand{\code}[1]{\texttt{#1}}

\title{Internal Architecture and Type--Checking Algorithms of Contemporary Proof Assistants}
\author{}
\date{}

\begin{document}
\maketitle
\vspace{-2em} % no introduction, jump straight in

\section{Layered Architecture Overview}
Contemporary proof assistants adopt a stratified design that separates a \emph{tiny, trusted kernel} from progressively less–trusted outer layers.  Fig.~\ref{fig:layers} summarises the canonical stack used by Coq, Lean 4, Agda, Isabelle/HOL and HOL Light\cite{coq-arch,isabelle-manual,hollight-tut,lean4-survey,agda-note}.  

\begin{figure}[h]
\centering
\begin{tabular}{|c|}
\hline
\textbf{User Interfaces} (VS Code, Proof General, CoqIDE, etc.) \\\hline
\textbf{Elaborator / Front-End}\\
\hline
\textbf{Tactic Engine \& Automation}\\
\hline
\textbf{Kernel (Type Checker + Inference Rules)}\\
\hline
\textbf{Logical Foundations (CIC, HOL, MLTT, …)}\\
\hline
\end{tabular}
\caption{Standard proof-assistant layer cake.}
\label{fig:layers}
\end{figure}

\section{Kernels in Practice}
\subsection{Size and Trusted Computing Base}
Each system’s kernel implements only the primitive inference rules of its object logic, plus definitional equality.  
Coq’s kernel (\(\approx\)6 kLoC OCaml) checks elaborated \emph{proof terms} for well-typedness in the Calculus of Inductive Constructions (CIC)\cite{coq-arch}.  
Lean 4’s C++ kernel is slightly larger (\(\approx\)9.5 kLoC) but mirrors the same minimalist philosophy\cite{lean4-survey}.  
HOL Light, in contrast, fits its entire kernel into \(\approx\)2.3 kLoC of OCaml by exploiting HOL’s small rule set\cite{hollight-tut}.  

\subsection{De Bruijn Criterion}
All surveyed assistants satisfy the \emph{de Bruijn criterion}: every externally supplied proof is reduced to kernel-checked primitives, guaranteeing that only the kernel must be trusted\cite{coq-core}.  Isabelle achieves this by embedding object logics inside a generic meta-logic implemented in Standard ML\cite{isabelle-manual}.  

\section{Type-Checking Algorithms}
\subsection{Bidirectional Checking}
Coq, Lean 4 and Agda implement \emph{bidirectional} algorithms that alternate between
\emph{checking} (\(\Gamma\vdash t:\!T\)) and \emph{inference} (\(\Gamma\vdash t\Rightarrow T\)) modes to localise where
conversion (\(\equiv\)) needs to be solved\cite{bidirectional-cic,tc-cic}.  
Lean 4 pushes most reductions eagerly (strong head β-normalisation) to minimise expensive definalisation at the leaves\cite{lean4-survey,lean4lean}.  

\subsection{Conversion and Normalisation}
All kernels rely on decidable conversion: two terms are definitionally equal iff their
normal forms are syntactically identical under β (and sometimes ι, ζ, δ) reduction.  
Coq uses a weak-head reduction with on-the-fly unfolding of transparent constants;  
Agda employs \emph{proof-irrelevance annotations} to erase compile-time proofs before equality checking, thereby accelerating normalisation\cite{agda-note}.  

\subsection{Universe Management}
Lean 4 and Coq feature cumulative universe hierarchies.  Lean’s algorithm stores
constraints in a union–find structure and relies on Tarjan’s algorithm for \(\le\)–closure\cite{lean4-survey}.  
Agda instead treats universe levels as first-class terms subject to unification, simplifying metaprogramming at the cost of heavier constraints\cite{agda-guide}.  

\section{Term Representation}
\paragraph{Names.}  Coq, Lean and Agda all compile surface names to \emph{de Bruijn indices},
ensuring α-equivalent terms share a binary encoding\cite{coq-core,lean4-survey}.  
Isabelle and HOL Light keep explicit identifiers because their HOL core lacks binding-sensitive rules\cite{isabelle-manual,hollight-tut}.  

\paragraph{Hash-Consing.}  Lean 4 hashes every node and maintains pointer equality to allow \(O(1)\) conversion checks on already-normalised sub-terms\cite{lean4-survey}.  HOL Light uses a related pointer-tagging trick for quick syntactic comparisons\cite{hollight-tut}.  

\section{Elaboration Front-Ends}
The elaborator translates user syntax (with holes, overloading, implicit arguments) into fully explicit kernel terms.
Coq’s \code{Vernac} language feeds an OCaml elaborator that performs first-order unification followed by metavariable resolution\cite{coq-arch}.  
Lean 4’s elaborator is written partly in Lean itself and exploits reflective tactics; it is intentionally \emph{not} in the trusted base\cite{lean4lean}.  
Agda’s interactive mode allows partially written programs; its elaborator inserts `?`-holes and later solves them by constraint propagation\cite{agda-guide}.  

\section{Automation and Tactics}
While tactics are outside the trusted core, their output is merely a proof term later verified by the kernel, preserving soundness\cite{coq-arch}.  
Isabelle/Isar uses a declarative proof language whose statements are compiled into kernel inferences\cite{isabelle-manual}.  

\section{Persistence and Compilation}
Coq compiles verified libraries into \code{.vo} objects that cache the normalised term and universe constraints to accelerate later replay; Lean 4 analogously stores \code{.olean} files\cite{lean4-survey}.  
HOL Light relies on OCaml’s marshalled values, whereas Isabelle uses poly/ML heaps\cite{hollight-tut,isabelle-manual}.  

\section{Verified Kernels}
Research projects like \emph{Candle}, a CakeML-verified re-implementation of HOL Light, show that entire kernels can be machine-checked down to machine code\cite{candle}.  MetaCoq and Lean4Lean pursue analogous self-verification for CIC and Lean respectively\cite{metacoq,bidirectional-cic,lean4lean}.  

\section{Conclusion (omitted)}
% No conclusion per user request
\bibliographystyle{plain}
\begin{thebibliography}{10}

\bibitem{coq-arch}
C.~Paulin-Mohring.
\newblock \emph{Introduction to the Coq Proof Assistant} (Course notes, 2025). [1]

\bibitem{coq-core}
Coq development team.
\newblock \emph{Core Language Reference Manual}, version 8.18.0. [2]

\bibitem{isabelle-manual}
T.~Nipkow, L.~C. Paulson, et al.
\newblock \emph{Isabelle System Manual}. [3][4]

\bibitem{hollight-tut}
J.~Harrison.
\newblock \emph{HOL Light Tutorial}. [5]

\bibitem{lean4-survey}
X.~Tang.
\newblock A Comprehensive Survey of the Lean 4 Theorem Prover. \emph{arXiv 2501.18639}, 2025. [6]

\bibitem{lean4lean}
M.~Carneiro.
\newblock Lean4Lean: a Lean 4 kernel in Lean. GitHub repository, 2025. [22]

\bibitem{agda-note}
Q.~Pu.
\newblock A Note on the Agda Codebase—Type Checking \& Reflection. HackMD, 2023. [8]

\bibitem{agda-guide}
Agda team.
\newblock Quick Guide to Editing, Type Checking and Compiling Agda Code, 2022. [11]

\bibitem{bidirectional-cic}
M.~Lennon-Bertrand.
\newblock Complete Bidirectional Typing for the Calculus of Inductive Constructions. \emph{ITP 2021}. [9]

\bibitem{tc-cic}
R.~Pollack.
\newblock Type Checking in Pure Type Systems. Technical Report, 1994. [12]

\bibitem{candle}
A.~Kumar et al.
\newblock Candle: A Verified Implementation of HOL Light. \emph{ITP 2022}. [10]

\bibitem{metacoq}
A.~Anand et al.
\newblock MetaCoq project: certifying Coq’s metatheory. Project page, 2018. [9]

\end{thebibliography}
\end{document}
