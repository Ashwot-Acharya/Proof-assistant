\documentclass[12pt]{article}

% ======== Packages =========
\usepackage[utf8]{inputenc}
\usepackage{amsmath, amssymb}
\usepackage{graphicx}
\usepackage{hyperref}
\usepackage{cite}
\usepackage{geometry}
\geometry{a4paper, margin=1in}
\usepackage{titlesec}
\titleformat{\section}{\large\bfseries}{\thesection}{1em}{}
\titleformat{\subsection}{\normalsize\bfseries}{\thesubsection}{1em}{}

\title{Working Principles of Proof Assistants and Formalization of some proofs in Agda}
\author{Bishesh Bohora, Supreme Chaudhary, Ashwot Acharya \\
\small Kathmandu University 
\small \texttt{}
}
%mention supervisor
\date{\today}

\begin{document}

\maketitle

\begin{abstract}
\end{abstract}

\tableofcontents
\newpage

\section{Introduction}

\section{Foundations}

\subsection{Logic Foundations}
This works assumes prior knowledge of Propositional and Predicate Logic. 
\subsubsection{Natural Deduction}
The propositions or formulas in Propositional Logic can be verified or proved simply by constructing their truth tables. But for logically complex propositions or propositions with many atomic statements, it becomes difficult to construct a truth table. With predicates, this becomes impossible. Therefore, to mitigate this we adhere to a basic set of inference rules with which we derive conclusions from assumptions in step by step, structured manner. The rule based system which allows us to reason about logical structure of propositions is known as \textbf{Natural Deduction}. \\
With the rules in (refer sth, either appendix or a book) we now present an example on how a proof is carried out,
(write a proof here).(also use law of excluded middle here)

The above works for propositional case, now adding these  Introduction and Elimination rules for quantifiers, we get a set of rules enough for predicate logic.
(write extra rules) as in (refer)

and a proof as (proof with quantifiers) 

The \textbf{soundness} and \textbf{completeness} of this system are discussed here (refer). 
\subsubsection{Intuitionstic Logic}




\section{Main Content}
Divide into multiple subsections as appropriate. For example:
\subsection{Topic A}
Discuss and explain this sub-topic clearly.

\subsection{Topic B}
Likewise, expand on another subtopic. Include figures, equations, or examples to aid exposition.

\section{Comparative Discussion}
(Optional) If your report compares different approaches, schools of thought, or perspectives, lay that out here.

\section{Applications or Case Studies}
(Optional) Describe practical uses or illustrative examples, if any.

\section{Summary and Conclusion}
Summarize the key points discussed. Mention what the reader should take away, and possibly directions for further reading or study.

% ======== References =========
\bibliographystyle{plain} % Or your preferred style
\bibliography{references}

\end{document}
