\documentclass{beamer}

\usetheme{Berlin}  

\usepackage[utf8]{inputenc}
\usepackage{graphicx}  
\usepackage{amsmath}   
\usepackage{hyperref}  
\usepackage{url}       
\usepackage{cite}        


\title{Working Principles of Proof Assitants and Formalization of Some Proofs in Agda}
\author{Ashwot Acharya, Bishesh Bohora, Supreme Chaudary}
\institute{Kathmandu University}
\date{\today} 


\begin{document}

\begin{frame}
  \titlepage
\end{frame}

\begin{frame}{Outline}
  \tableofcontents
\end{frame}

\section{Introduction And Motivation}

\begin{frame}{Introduction}
    As the title suggests, our project will revolve around exploration of theoritical fondations behind Proof Assitants and practice them.

    
\end{frame}

\begin{frame}{Proof Assistants}
        In computer science and mathematical logic, a proof assistant or interactive theorem prover is a software tool to assist with the development of formal proofs by human–machine collaboration.\cite{wikipedia_proof_assistant} 

        Examples:
        \begin{itemize}
            \item Coq
            \item LEAN 
            \item Agda
        \end{itemize}
\end{frame}
\begin{frame}{History}
\begin{itemize}
  \item \textbf{Gödel’s Incompleteness Theorems (1930s)}: Revealed limitations of formal systems; sparked interest in formal logic and verification.

  \item \textbf{ Computability Theory (1940s–50s)}: Turing machines and $\lambda$-calculus laid the groundwork for mechanized reasoning.
\item \textbf{Logic Theorist (1954)}: First automated theorem prover by Newell and Simon, capable of proving theorems in propositional logic.

  \item \textbf{LISP (1960)}: A symbolic programming language created by John McCarthy; became essential for early theorem proving systems.

   \item \textbf{Automath (1967)}: First system to check mathematical proofs using dependent types.
 
  
\end{itemize}
\end{frame} 
\begin{frame}

\begin{itemize}
   \item \textbf{LCF \& ML (1970s)}: Introduced tactic-based proofs and the ML programming language; foundational to later systems.
  
  \item \textbf{Coq (1986)}: A proof assistant based on constructive type theory, supporting verified programming and formal proofs.
  
  \item \textbf{Isabelle (1989)}: Generic theorem prover with support for multiple logics and strong automation tools.

  \item \textbf{Four-Color Theorem (1996)}: First major mathematical theorem re-verified by proof assistants (Coq and HOL).
  
  \item \textbf{Feit-Thompson Theorem (2012)}: Large-scale group theory proof formalized in Coq, showcasing proof assistant capability.
  
  \item \textbf{Lean (2015-2023)}: Modern proof assistant combining type theory with performance and usability; popular in formal math via \texttt{mathlib}.
  
\end{itemize}
\end{frame}
\begin{frame}{Motivation}
    \begin{itemize}
  \item Strong interest in mathematics and formal reasoning.
  \item Discovered type theory through internet memes on category theory.
  \item Fascinated by the Four Colour Theorem and its computer-assisted proof.
  \item The rise of AI raised the question: \textit{"How do computers understand reasoning?"}
  \item Drawn to functional programming, which closely mirrors mathematical logic and structure.
    
\end{itemize}

\end{frame}



\section{Glance At Theoritical Foundations}
\begin{frame}{Type Theory}
\begin{itemize}
  \item Type theory is a formal system that classifies expressions by their "types."
  \item Originally developed as an alternative to set theory for foundations of mathematics.
  \item Predecessor to Dependent Type Theory, Martin Löf Type Theory which form basis for various proof assitants.
  \item Types prevent logical paradoxes and provide a basis for constructive reasoning.
\end{itemize}
\end{frame}
\begin{frame}{Curry–Howard Correspondence}
\begin{itemize}
  \item A deep analogy between **logic and computation**:
  \begin{itemize}
    \item Propositions $\leftrightarrow$ Types
    \item Proofs $\leftrightarrow$ Programs
  \end{itemize}
  \item A proof of a proposition is a program of a corresponding type.
  \item Enables writing code that is **correct-by-construction**.
  \item Fundamental to systems like Coq, where proving a theorem is like writing a program.
\end{itemize}
\end{frame}
\begin{frame}{$\lambda$-Calculus and Functional Programming}
    \begin{itemize}
        \item \textbf{Lambda Calculus}: A minimal formal system for function definition and application; the foundation of computation theory.
  \item \textbf{Functional Programming}: Directly inspired by lambda calculus; treats computation as evaluation of mathematical functions.

  \item In proof assitants, Core logic is based on typed lambda calculus.
      \item Tools like Coq and Agda embed functional languages with type theory.
    \end{itemize}
\end{frame}

\section{Methodology and Tool}
\begin{frame}{Methodology}
    
\end{frame}
\begin{frame}{Agda}
Agda is a functional programming language with dependent types. It is based on Martin Löf Type Theory. And most importantly it is a proof assitant.
\cite{inproceedings}
\end{frame}
\begin{frame}{Why Agda?}
\end{frame}

\section{Work Plan}
\begin{frame}{Work Plan}
\end{frame}

\section{Significance and Expected Outcomes}
\begin{frame}{Significance}
\end{frame}
\begin{frame}{Expected Outcomes}
\end{frame}

\section{Conclusion}
\begin{frame}{References}
\bibliographystyle{apalike}       
\bibliography{references} 
\end{frame}

       

\end{document}
